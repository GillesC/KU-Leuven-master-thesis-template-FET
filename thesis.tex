%%%%%%%%%%%%%%%%%%%%%%%%%%%%%%%%%%%%%%%%%%%%%%%%%%%%%%%%%%%%%%%%%%%%%%%%
%                                                                      %
% LaTeX, FIIW thesis template                                          %
% 14/03/2023 v1.3                                                      %
%                                                                      %
%%%%%%%%%%%%%%%%%%%%%%%%%%%%%%%%%%%%%%%%%%%%%%%%%%%%%%%%%%%%%%%%%%%%%%%%
\documentclass[11pt,a4paper]{report}
% Indien je je thesis recto-verso wil afdrukken gebruik je onderstaande opties i.p.v. bovenstaande
%\documentclass[11pt,a4paper,twoside,openright]{report}


\usepackage[a4paper,left=3.5cm, right=2.5cm, top=3.5cm, bottom=3.5cm]{geometry}
\usepackage{graphicx}
\graphicspath{{./figs/}}                % set graphics path to figs folder, ie now all file imports can be referenced relative to figs
\usepackage[latin1]{inputenc}           % om niet ascii karakters rechtstreeks te kunnen inputten
%\usepackage[utf8]{inputenc}            % commentarieer deze regel uit als je utf8 encoded files gebruikt in plaats van latin1
\usepackage[backend=biber, style=ieee, citestyle=numeric-comp, maxnames=99]{biblatex}
\addbibresource{bib.bib}
\AtBeginBibliography{\footnotesize}
\usepackage{cmbright} % new improved font
\usepackage{listings}             		% voor het weergeven van broncode
\usepackage[outputdir=cache]{minted}                     % for beautiful listings
\usemintedstyle{borland}
\usepackage{verbatim}					% weergeven van code, commando's, ...
\usepackage{hyperref}					% maak PDF van de thesis navigeerbaar
\usepackage{url}						% URL's invoegen in tekst met behulp van \url{http://}
\usepackage[small,bf,hang]{caption}     % om de captions wat te verbeteren
\usepackage[final]{pdfpages}            % gebruikt voor het invoegen van het artikel in pdf-formaat
%\usepackage{pslatex}					% andere lettertype's dan de standaard types

%\usepackage{sectsty}					% aanpassen van de fonts van sections en chapters
%\allsectionsfont{\sffamily}
%\chapterfont{\raggedleft\sffamily}

\usepackage{float}                      % De optie H voor de plaatsing van figuren op de plaats waar je ze invoegt. bvb. \begin{figure}[H]
%\usepackage{longtable}					% tabellen die over meerdere pagina's gespreid worden
%\usepackage[times]{quotchap}           % indien je fancy hoofdstuktitels wil
%\usepackage[none]{hyphenat}
%\usepackage{latexsym}
\usepackage{amsmath}
\usepackage{amssymb}
\usepackage{siunitx}
\sisetup{detect-all}
\usepackage[acronym,xindy]{glossaries}
\makenoidxglossaries
\usepackage[version=4]{mhchem} % chemical formulas
\usepackage{tabularx}
\usepackage{booktabs} % nice tables
\usepackage{array} % fixed-width columns in tables


%%%%%%%%%%%% choose your campus and language %%%%%%%%%%%%
% current options are: fiiw_gent (fiiw_ghent_eng), fiiw_denayer (fiiw_denayer_eng)
\usepackage{fiiw_gent} 
% \usepackage{fiiw_ghent_eng} % For the english version also change last page at the bottom of this file!



%door onderstaande regels in commentaar te zetten, of op false, kan je pagina's weglaten
%bijvoorbeeld het weglaten van een voorwoord, lijst met symbolen, ...
%%%%%%%%%%%%%%%%%%%%%%%%%%%%%%%%%%%%%%%%%%%%%%%%%%%%%%%%%%%%%%%%%%%%%%%%%%%%%%%%%%%%%%%%
%voorwoord toevoegen?
\acknowledgementspagetrue
\acknowledgements{voorwoord}			%.tex file met daarin het voorwoord
%abstract toevoegen?
\abstractpagetrue
\abstracts{abstract}					%.tex file met daarin het abstract
%lijst van figuren toevoegen?
%\listoffigurespagetrue
%lijst van tabellen toevoegen?
%\listoftablespagetrue
%lijst van symbolen toevoegen?
%\listofsymbolspagetrue
%\listofsymbols{symbolen}				%.tex file met daarin de lijst van symbolen


%informatie over het eindwerk, de promotor, ...
%%%%%%%%%%%%%%%%%%%%%%%%%%%%%%%%%%%%%%%%%%%%%%%
\opleiding{naam opleiding}
\afdeling{afstudeerrichting vermelden}

\title{Titel Masterproef}
\subtitle{Ondertitel (facultatief)}
% \author{naam student}
\forename{Voornaam}
\surname{Achternaam}
\academicyear{2013 - 2014}

\promotorA[Promotor(en)]{My Promotor}
\promotorB[Co-promotor(en)]{(My Co-promotor)\\(My company promotor)}



\newacronym{iot}{IoT}{Internet-of-Things} % default expression for defining a new acronym
\newacronym[plural=LPWANs,firstplural=Low-Power Wide-Area Networks (LPWANs)]{lpwan}{LPWAN}{Low-Power Wide-Area Network}

\begin{document}
\preface

\printnoidxglossary[type=\acronymtype]%
\clearpage

% !TeX root = thesis.tex


%Het eerste hoofdstuk van je thesis.
\chapter{Changes w.r.t. the default template}\label{ch:introduction}



\begin{itemize}
\item Changed font from default computer modern to CMBright.
\item Verhoogde baseline-skip in de titel. Zodat letters niet tegen elkaar plakken
\item Siunitx package auto toegevoegd zie gebruik: \url{https://github.com/dramco-edu/LaTex/wiki/LaTex-Tips-and-Tricks#using-si-units-numbers-angles-etc}
\item Gebruik nu formules inclusief eenheden via het commando:
\begin{lstlisting}
    \begin{equation}
     E = m c^2
     \tagaddtext{[\si{\joule}]}
    \end{equation}
    \end{lstlisting},
which outputs:
\begin{equation}\label{eq:emc2}
    E = m c^2%
    \tagaddtext{[\si{\joule}]}
\end{equation}
\item Afkortingen:
\begin{itemize}
    \item  \gls{iot}      % outputs (if first occurance): Internet-of-Things (IoT)
          % outputs (else): IoT
    \item \glspl{lpwan}  % outputs (if first occurance): Low-Power Wide-Area Networks (LPWANs)
          % outputs (else): LPWANs
    \item \acrlong{iot}  % outputs: Internet-of-Things
    \item \acrshort{iot} % outputs: IoT
    \item \acrfull{iot}  % outputs: Internet-of-Things (IoT)
\end{itemize}
\item \verb!\ce{CO2}! \ce{CO2}
\item Refs:
\begin{itemize}

    \item \cref{fig:example}
    \item \cref{tab:example}
    \item \cref{ch:introduction}
    \item \cref{eq:emc2}


    \item \verb!\cref{fig:example}! becomes \cref{fig:example}
    \item \verb!\cref{tab:example}! becomes \cref{tab:example}
    \item \verb!\cref{ch:introduction}! becomes \cref{ch:introduction}
    \item \cref{eq:emc2}


\end{itemize}
\item numerical citation~\cite{s19030585}
\begin{itemize}
    \item \verb!\cite{CallebautGilles2019ByoS}! becomes \cite{CallebautGilles2019ByoS}

    \item \verb!\citeauthor{CallebautGilles2019ByoS}! becomes \citeauthor{CallebautGilles2019ByoS}
    \item \verb!\citeauthorref{CallebautGilles2019ByoS}! becomes \citeauthorref{CallebautGilles2019ByoS}
    \item \verb!\autocite[chap.~2]{CallebautGilles2019ByoS}! becomes \autocite[chap.~2]{CallebautGilles2019ByoS}
    \item \verb!\footcite{CallebautGilles2019ByoS}! becomes \footcite{CallebautGilles2019ByoS}
    \item \verb!\fullcite{CallebautGilles2019ByoS}! becomes \fullcite{CallebautGilles2019ByoS}
\end{itemize}

    \item A $\SI{45}{\degree}$ angle or a \ang{45}.

    \item It is \SI{17}{\degreeCelsius} outside.

    \item  \num{10000}
    \item \num{3.45d-4}

    \item \si{\kilo\gram\meter\per\square\second} 
    \item \si{kg.m/s^2} %unit only

    \item \SI{10}{\percent}
    \item \SI{68}{kg}
\end{itemize}


\begin{center}
    \begin{tabular}{ l  l  l  p{5cm}}
        \hline
        Day       & Min Temp & Max Temp & Summary                                                      \\ \hline
        Monday    & 11C      & 22C      & A clear day with lots of sunshine.
        However, the strong breeze will bring down the temperatures.                                   \\ \hline
        Tuesday   & 9C       & 19C      & Cloudy with rain, across many northern regions. Clear spells
        across most of Scotland and Northern Ireland,
        but rain reaching the far northwest.                                                           \\ \hline
        Wednesday & 10C      & 21C      & Rain will still linger for the morning.
        Conditions will improve by early afternoon and continue
        throughout the evening.                                                                        \\
        \hline
    \end{tabular}
\end{center}







\newcolumntype{R}{>{\raggedleft\arraybackslash}X}%


\begin{center}
    \begin{tabularx}{\textwidth}{ cccX }
        \hline
        Day       & Min Temp & Max Temp & Summary                                                      \\ \hline
        Monday    & 11C      & 22C      & A clear day with lots of sunshine.
        However, the strong breeze will bring down the temperatures.                                   \\ \hline
        Tuesday   & 9C       & 19C      & Cloudy with rain, across many northern regions. Clear spells
        across most of Scotland and Northern Ireland,
        but rain reaching the far northwest.                                                           \\ \hline
        Wednesday & 10C      & 21C      & Rain will still linger for the morning.
        Conditions will improve by early afternoon and continue
        throughout the evening.                                                                        \\
        \hline
    \end{tabularx}
\end{center}



\begin{figure}
    \centering
    \includegraphics[width=0.95\linewidth]{example.jpg}
    \caption{Example JPG}%
    \label{fig:example}
\end{figure}

\begin{table}[ht]
    \centering
    \caption{Fixed-width columns.}%
    \label{tab:example}
    \begin{tabular}[t]{l>{\raggedright}p{0.3\linewidth}>{\raggedright\arraybackslash}p{0.3\linewidth}}
        \toprule
                     & Treatment A                    & Treatment B                     \\
        \midrule
        John Smith   & Good response, no side-effects & No response                     \\
        Jane Doe     & --                             & Good response, no side-effects  \\
        Mary Johnson & No response                    & Good response with side-effects \\
        \bottomrule
    \end{tabular}
\end{table}%



\begin{listing}[ht]
    \inputminted{python}{code/example.py}
    \caption{Minimal working example}
    \label{listing:1}
\end{listing}



%% !TeX root = thesis.tex


%Het eerste hoofdstuk van je thesis.
\chapter{Changes w.r.t. the default template}\label{ch:introduction}



\begin{itemize}
\item Changed font from default computer modern to CMBright.
\item Verhoogde baseline-skip in de titel. Zodat letters niet tegen elkaar plakken
\item Siunitx package auto toegevoegd zie gebruik: \url{https://github.com/dramco-edu/LaTex/wiki/LaTex-Tips-and-Tricks#using-si-units-numbers-angles-etc}
\item Gebruik nu formules inclusief eenheden via het commando:
\begin{lstlisting}
    \begin{equation}
     E = m c^2
     \tagaddtext{[\si{\joule}]}
    \end{equation}
    \end{lstlisting},
which outputs:
\begin{equation}\label{eq:emc2}
    E = m c^2%
    \tagaddtext{[\si{\joule}]}
\end{equation}
\item Afkortingen:
\begin{itemize}
    \item  \gls{iot}      % outputs (if first occurance): Internet-of-Things (IoT)
          % outputs (else): IoT
    \item \glspl{lpwan}  % outputs (if first occurance): Low-Power Wide-Area Networks (LPWANs)
          % outputs (else): LPWANs
    \item \acrlong{iot}  % outputs: Internet-of-Things
    \item \acrshort{iot} % outputs: IoT
    \item \acrfull{iot}  % outputs: Internet-of-Things (IoT)
\end{itemize}
\item \verb!\ce{CO2}! \ce{CO2}
\item Refs:
\begin{itemize}

    \item \cref{fig:example}
    \item \cref{tab:example}
    \item \cref{ch:introduction}
    \item \cref{eq:emc2}


    \item \verb!\cref{fig:example}! becomes \cref{fig:example}
    \item \verb!\cref{tab:example}! becomes \cref{tab:example}
    \item \verb!\cref{ch:introduction}! becomes \cref{ch:introduction}
    \item \cref{eq:emc2}


\end{itemize}
\item numerical citation~\cite{s19030585}
\begin{itemize}
    \item \verb!\cite{CallebautGilles2019ByoS}! becomes \cite{CallebautGilles2019ByoS}

    \item \verb!\citeauthor{CallebautGilles2019ByoS}! becomes \citeauthor{CallebautGilles2019ByoS}
    \item \verb!\citeauthorref{CallebautGilles2019ByoS}! becomes \citeauthorref{CallebautGilles2019ByoS}
    \item \verb!\autocite[chap.~2]{CallebautGilles2019ByoS}! becomes \autocite[chap.~2]{CallebautGilles2019ByoS}
    \item \verb!\footcite{CallebautGilles2019ByoS}! becomes \footcite{CallebautGilles2019ByoS}
    \item \verb!\fullcite{CallebautGilles2019ByoS}! becomes \fullcite{CallebautGilles2019ByoS}
\end{itemize}

    \item A $\SI{45}{\degree}$ angle or a \ang{45}.

    \item It is \SI{17}{\degreeCelsius} outside.

    \item  \num{10000}
    \item \num{3.45d-4}

    \item \si{\kilo\gram\meter\per\square\second} 
    \item \si{kg.m/s^2} %unit only

    \item \SI{10}{\percent}
    \item \SI{68}{kg}
\end{itemize}


\begin{center}
    \begin{tabular}{ l  l  l  p{5cm}}
        \hline
        Day       & Min Temp & Max Temp & Summary                                                      \\ \hline
        Monday    & 11C      & 22C      & A clear day with lots of sunshine.
        However, the strong breeze will bring down the temperatures.                                   \\ \hline
        Tuesday   & 9C       & 19C      & Cloudy with rain, across many northern regions. Clear spells
        across most of Scotland and Northern Ireland,
        but rain reaching the far northwest.                                                           \\ \hline
        Wednesday & 10C      & 21C      & Rain will still linger for the morning.
        Conditions will improve by early afternoon and continue
        throughout the evening.                                                                        \\
        \hline
    \end{tabular}
\end{center}







\newcolumntype{R}{>{\raggedleft\arraybackslash}X}%


\begin{center}
    \begin{tabularx}{\textwidth}{ cccX }
        \hline
        Day       & Min Temp & Max Temp & Summary                                                      \\ \hline
        Monday    & 11C      & 22C      & A clear day with lots of sunshine.
        However, the strong breeze will bring down the temperatures.                                   \\ \hline
        Tuesday   & 9C       & 19C      & Cloudy with rain, across many northern regions. Clear spells
        across most of Scotland and Northern Ireland,
        but rain reaching the far northwest.                                                           \\ \hline
        Wednesday & 10C      & 21C      & Rain will still linger for the morning.
        Conditions will improve by early afternoon and continue
        throughout the evening.                                                                        \\
        \hline
    \end{tabularx}
\end{center}



\begin{figure}
    \centering
    \includegraphics[width=0.95\linewidth]{example.jpg}
    \caption{Example JPG}%
    \label{fig:example}
\end{figure}

\begin{table}[ht]
    \centering
    \caption{Fixed-width columns.}%
    \label{tab:example}
    \begin{tabular}[t]{l>{\raggedright}p{0.3\linewidth}>{\raggedright\arraybackslash}p{0.3\linewidth}}
        \toprule
                     & Treatment A                    & Treatment B                     \\
        \midrule
        John Smith   & Good response, no side-effects & No response                     \\
        Jane Doe     & --                             & Good response, no side-effects  \\
        Mary Johnson & No response                    & Good response with side-effects \\
        \bottomrule
    \end{tabular}
\end{table}%



\begin{listing}[ht]
    \inputminted{python}{code/example.py}
    \caption{Minimal working example}
    \label{listing:1}
\end{listing}



%% !TeX root = thesis.tex

%second chapter of your thesis
\chapter{Bespreking}
In het vorig hoofdstuk hebben we naar deze tekst verwezen\label{verwijzing}.

%% !TeX root = thesis.tex

%%%%%%%%%%%%%%%%%%%%%%%%%%%%%%%%%%%%%%%%%%%%%%%%%%%%%%%%%%%%%%%%%%% 
%                                                                 %
%                            CHAPTER                              %
%                                                                 %
%%%%%%%%%%%%%%%%%%%%%%%%%%%%%%%%%%%%%%%%%%%%%%%%%%%%%%%%%%%%%%%%%%% 
 
\chapter{This is the another Chapter}
 
You can say great work has been done about something \citep{Castleman98,Granlund95} or say that \citet{Holmes95} did something really great.
xxxx xxxxx xxxx xxxxxxxxx 
xxx xxxxx xxxxx xxx xxxx xxxx xxxxx xxxxx xxxx xxxxx xxxx xxxxxxxxx
 
\begin{figure}
\vspace{2.0in}
\caption{This is the Caption for Figure 1}
\end{figure}
 
xxx xxxxx xxxxx xxx xxxx xxxx xxxxx xxxxx xxxx xxxxx xxxx xxxxxxxxx
xxx xxxxx xxxxx xxx xxxx xxxx xxxxx xxxxx xxxx xxxxx xxxx xxxxxxxxx
xxx xxxxx xxxxx xxx xxxx xxxx xxxxx xxxxx xxxx xxxxx xxxx xxxxxxxxx
 
\begin{table}[t]
\begin{center}
\begin{tabular}{lll}
Here's       & an          & example  \\
of           & a           & table    \\
floated      & with        & the      \\
\verb+table+ & environment & command.
\end{tabular}
\end{center}
\caption{This is the Caption for Table 1}
\end{table}
 
xxx xxxxx xxxxx xxx xxxx xxxx xxxxx xxxxx xxxx xxxxx xxxx xxxxxxxxx
xxx xxxxx xxxxx xxx xxxx xxxx xxxxx xxxxx xxxx xxxxx xxxx xxxxxxxxx
 
\section{This is a Section Heading}
 

% Bibliografie: referenties. De items zitten in bib.bib
%%%%%%%%%%%%%%%%%%%%%%%%%%%%%%%%%%%%%%%%%%%%%%%%%%%%%%%%%%%%%%%%%
\printbibliography


% Eventueel enkele appendices
%%%%%%%%%%%%%%%%%%%%%%%%%%%%%%
\appendix
\input{bijlage1}

% Bijlage met daarin het wetenschappelijk artikel
%%%%%%%%%%%%%%%%%%%%%%%%%%%%%%%%%%%%%%%%%%%%%%%%%%
\chapter{Beschrijving van deze masterproef in de vorm van een wetenschappelijk artikel}
%\includepdf{artikel.pdf}

% Bijlage met daarin de poster
%%%%%%%%%%%%%%%%%%%%%%%%%%%%%%%
\chapter{Poster}
%\includepdf{poster.pdf}

%\includepdf{back_fiiw_gent.pdf}
% \includepdf{back_fiiw_ghent_eng.pdf} % For the english version

\end{document}
