%%%%%%%%%%%%%%%%%%%%%%%%%%%%%%%%%%%%%%%%%%%%%%%%%%%%%%%%%%%%%%%%%%%%%%%%
%                                                                      %
% LaTeX, FIIW thesis template                                          %
% 19/12/2023 v1.3                                                      %
%  TODOs:                                                              %
%   - Translate all comments to English                                %
%%%%%%%%%%%%%%%%%%%%%%%%%%%%%%%%%%%%%%%%%%%%%%%%%%%%%%%%%%%%%%%%%%%%%%%%


\documentclass[11pt,a4paper,twoside,openright]{report}
% Indien je je thesis recto-verso wil afdrukken gebruik je onderstaande opties i.p.v. bovenstaande
%\documentclass[11pt,a4paper,twoside,openright]{report}
\usepackage[a4paper,left=3.5cm, right=2.5cm, top=3.5cm, bottom=3.5cm]{geometry}
\usepackage{graphicx}
\graphicspath{{./figs/}}                        % set graphics path to figs folder, ie now all file imports can be referenced relative to figs
\usepackage[latin1]{inputenc}                   % om niet ascii karakters rechtstreeks te kunnen inputten
%\usepackage[utf8]{inputenc}                    % commentarieer deze regel uit als je utf8 encoded files gebruikt in plaats van latin1
\usepackage[backend=biber, style=ieee, 
citestyle=numeric-comp, maxnames=99]{biblatex}  % make use of the biblatex package to cite references
\addbibresource{bib.bib}
\AtBeginBibliography{\footnotesize}

\usepackage{cmbright}                           % new improved font
\usepackage{listings}             		        % voor het weergeven van broncode
\usepackage[outputdir=cache]{minted}                     % for beautiful listings
\usemintedstyle{borland}
\usepackage{verbatim}					        % weergeven van code, commando's, ...
\usepackage{hyperref}					        % maak PDF van de thesis navigeerbaar
\usepackage{url}						        % URL's invoegen in tekst met behulp van \url{http://}
\usepackage[small,bf,hang]{caption}             % om de captions wat te verbeteren
\usepackage[final]{pdfpages}                    % gebruikt voor het invoegen van het artikel in pdf-formaat
%\usepackage{pslatex}					        % andere lettertype's dan de standaard types

%\usepackage{sectsty}					        % aanpassen van de fonts van sections en chapters
%\allsectionsfont{\sffamily}
%\chapterfont{\raggedleft\sffamily}

\usepackage{float}                              % De optie H voor de plaatsing van figuren op de plaats waar je ze invoegt. bvb. \begin{figure}[H]
%\usepackage{longtable}					        % tabellen die over meerdere pagina's gespreid worden
%\usepackage[times]{quotchap}                   % indien je fancy hoofdstuktitels wil
%\usepackage[none]{hyphenat}
%\usepackage{latexsym}
\usepackage{amsmath}
\usepackage{amssymb}
\usepackage{siunitx}
\sisetup{detect-all}
\usepackage[acronym,xindy]{glossaries}
\makenoidxglossaries
\usepackage[version=4]{mhchem}                  % chemical formulas
\usepackage{tabularx}
\usepackage{booktabs}                           % nice tables
\usepackage{array}                              % fixed-width columns in tables

%%%% Tikz %%%%
\usepackage{pgfplots}
\DeclareUnicodeCharacter{2212}{−}
\usepgfplotslibrary{groupplots,dateplot}
\usetikzlibrary{patterns,shapes.arrows}
\pgfplotsset{compat=newest}
%%%%%%%%%%%% MAKE FIGURES MORE UNIFORM %%%%%%%%%%%%
\definecolor{darkgray176}{RGB}{176,176,176}
\definecolor{color0}{RGB}{255,127,14}
\definecolor{color1}{RGB}{44,160,44}
\pgfplotsset{
    every axis/.append  style={
        title style={draw=none},
        label style={font=\small},
        legend style={
            fill opacity=0.8,
            nodes={scale=0.8, transform shape}, {draw=none}
        },
        tick align=outside,
        tick pos=left,
        x grid style={darkgray176},
        xtick style={color=black},
        y grid style={darkgray176},
        ytick style={color=black},
        grid=both,
    },
    every axis plot/.append style={
        line width=1.0pt,
        mark size=1,
    },
}



%%%%%%%%%%%% choose your campus and language %%%%%%%%%%%%
\usepackage{fiiw} 



%door onderstaande regels in commentaar te zetten, of op false, kan je pagina's weglaten
%bijvoorbeeld het weglaten van een voorwoord, lijst met symbolen, ...
%%%%%%%%%%%%%%%%%%%%%%%%%%%%%%%%%%%%%%%%%%%%%%%%%%%%%%%%%%%%%%%%%%%%%%%%%%%%%%%%%%%%%%%%
%voorwoord toevoegen?
\acknowledgementspagetrue
\acknowledgements{preface}			%.tex file met daarin het voorwoord
%abstract toevoegen?
\abstractpagetrue
\abstracts{abstract}					%.tex file met daarin het abstract
%lijst van figuren toevoegen?
%\listoffigurespagetrue
%lijst van tabellen toevoegen?
%\listoftablespagetrue
%lijst van symbolen toevoegen?
%\listofsymbolspagetrue
%\listofsymbols{symbolen}				%.tex file met daarin de lijst van symbolen


% Information about your discipline
%%%%%%%%%%%%%%%%%%%%%%%%%%%%%%%%%%%%%%%%%%%%%%%

\opleiding{discipline name}
\afdeling{specialisation}
\campus{ghenteng}                       % Define your campus and language (append "eng" to load the English template)
                                            % campuses: denayer, geel, gent, groept, brugge (denayereng, geeleng, ghenteng, groupteng, brugeseng)
\title{Title Masterproef}
\subtitle{Subtitle (optional)}
\forenameA{first name}
\surnameA{last name}
\forenameB{} %keep empty if no 2nd author
\surnameB{} %keep empty if no 2nd author
\academicyear{2024 - 2025}


\promotorA[Promotor(en)]{My Promotor} %for English use Supervisor(s)
\promotorB[Co-promotor(en)]{My Co-promotor\\My company promotor}

\loadLanguage% keep this untouched



\newacronym{iot}{IoT}{Internet-of-Things} % default expression for defining a new acronym
\newacronym[plural=LPWANs,firstplural=Low-Power Wide-Area Networks (LPWANs)]{lpwan}{LPWAN}{Low-Power Wide-Area Network}

\begin{document}
\preface%

\printnoidxglossary[type=\acronymtype]%
\clearpage

% !TeX root = thesis.tex


%Het eerste hoofdstuk van je thesis.
\chapter{Changes w.r.t. the default template}\label{ch:introduction}



\begin{itemize}
\item Changed font from default computer modern to CMBright.
\item Verhoogde baseline-skip in de titel. Zodat letters niet tegen elkaar plakken
\item Siunitx package auto toegevoegd zie gebruik: \url{https://github.com/dramco-edu/LaTex/wiki/LaTex-Tips-and-Tricks#using-si-units-numbers-angles-etc}
\item Gebruik nu formules inclusief eenheden via het commando:
\begin{lstlisting}
    \begin{equation}
     E = m c^2
     \tagaddtext{[\si{\joule}]}
    \end{equation}
    \end{lstlisting},
which outputs:
\begin{equation}\label{eq:emc2}
    E = m c^2%
    \tagaddtext{[\si{\joule}]}
\end{equation}
\item Afkortingen:
\begin{itemize}
    \item  \gls{iot}      % outputs (if first occurance): Internet-of-Things (IoT)
          % outputs (else): IoT
    \item \glspl{lpwan}  % outputs (if first occurance): Low-Power Wide-Area Networks (LPWANs)
          % outputs (else): LPWANs
    \item \acrlong{iot}  % outputs: Internet-of-Things
    \item \acrshort{iot} % outputs: IoT
    \item \acrfull{iot}  % outputs: Internet-of-Things (IoT)
\end{itemize}
\item \verb!\ce{CO2}! \ce{CO2}
\item Refs:
\begin{itemize}

    \item \cref{fig:example}
    \item \cref{tab:example}
    \item \cref{ch:introduction}
    \item \cref{eq:emc2}


    \item \verb!\cref{fig:example}! becomes \cref{fig:example}
    \item \verb!\cref{tab:example}! becomes \cref{tab:example}
    \item \verb!\cref{ch:introduction}! becomes \cref{ch:introduction}
    \item \cref{eq:emc2}


\end{itemize}
\item numerical citation~\cite{s19030585}
\begin{itemize}
    \item \verb!\cite{CallebautGilles2019ByoS}! becomes \cite{CallebautGilles2019ByoS}

    \item \verb!\citeauthor{CallebautGilles2019ByoS}! becomes \citeauthor{CallebautGilles2019ByoS}
    \item \verb!\citeauthorref{CallebautGilles2019ByoS}! becomes \citeauthorref{CallebautGilles2019ByoS}
    \item \verb!\autocite[chap.~2]{CallebautGilles2019ByoS}! becomes \autocite[chap.~2]{CallebautGilles2019ByoS}
    \item \verb!\footcite{CallebautGilles2019ByoS}! becomes \footcite{CallebautGilles2019ByoS}
    \item \verb!\fullcite{CallebautGilles2019ByoS}! becomes \fullcite{CallebautGilles2019ByoS}
\end{itemize}

    \item A $\SI{45}{\degree}$ angle or a \ang{45}.

    \item It is \SI{17}{\degreeCelsius} outside.

    \item  \num{10000}
    \item \num{3.45d-4}

    \item \si{\kilo\gram\meter\per\square\second} 
    \item \si{kg.m/s^2} %unit only

    \item \SI{10}{\percent}
    \item \SI{68}{kg}
\end{itemize}


\begin{center}
    \begin{tabular}{ l  l  l  p{5cm}}
        \hline
        Day       & Min Temp & Max Temp & Summary                                                      \\ \hline
        Monday    & 11C      & 22C      & A clear day with lots of sunshine.
        However, the strong breeze will bring down the temperatures.                                   \\ \hline
        Tuesday   & 9C       & 19C      & Cloudy with rain, across many northern regions. Clear spells
        across most of Scotland and Northern Ireland,
        but rain reaching the far northwest.                                                           \\ \hline
        Wednesday & 10C      & 21C      & Rain will still linger for the morning.
        Conditions will improve by early afternoon and continue
        throughout the evening.                                                                        \\
        \hline
    \end{tabular}
\end{center}







\newcolumntype{R}{>{\raggedleft\arraybackslash}X}%


\begin{center}
    \begin{tabularx}{\textwidth}{ cccX }
        \hline
        Day       & Min Temp & Max Temp & Summary                                                      \\ \hline
        Monday    & 11C      & 22C      & A clear day with lots of sunshine.
        However, the strong breeze will bring down the temperatures.                                   \\ \hline
        Tuesday   & 9C       & 19C      & Cloudy with rain, across many northern regions. Clear spells
        across most of Scotland and Northern Ireland,
        but rain reaching the far northwest.                                                           \\ \hline
        Wednesday & 10C      & 21C      & Rain will still linger for the morning.
        Conditions will improve by early afternoon and continue
        throughout the evening.                                                                        \\
        \hline
    \end{tabularx}
\end{center}



\begin{figure}
    \centering
    \includegraphics[width=0.95\linewidth]{example.jpg}
    \caption{Example JPG}%
    \label{fig:example}
\end{figure}

\begin{table}[ht]
    \centering
    \caption{Fixed-width columns.}%
    \label{tab:example}
    \begin{tabular}[t]{l>{\raggedright}p{0.3\linewidth}>{\raggedright\arraybackslash}p{0.3\linewidth}}
        \toprule
                     & Treatment A                    & Treatment B                     \\
        \midrule
        John Smith   & Good response, no side-effects & No response                     \\
        Jane Doe     & --                             & Good response, no side-effects  \\
        Mary Johnson & No response                    & Good response with side-effects \\
        \bottomrule
    \end{tabular}
\end{table}%



\begin{listing}[ht]
    \inputminted{python}{code/example.py}
    \caption{Minimal working example}
    \label{listing:1}
\end{listing}



%% !TeX root = thesis.tex


%Het eerste hoofdstuk van je thesis.
\chapter{Changes w.r.t. the default template}\label{ch:introduction}



\begin{itemize}
\item Changed font from default computer modern to CMBright.
\item Verhoogde baseline-skip in de titel. Zodat letters niet tegen elkaar plakken
\item Siunitx package auto toegevoegd zie gebruik: \url{https://github.com/dramco-edu/LaTex/wiki/LaTex-Tips-and-Tricks#using-si-units-numbers-angles-etc}
\item Gebruik nu formules inclusief eenheden via het commando:
\begin{lstlisting}
    \begin{equation}
     E = m c^2
     \tagaddtext{[\si{\joule}]}
    \end{equation}
    \end{lstlisting},
which outputs:
\begin{equation}\label{eq:emc2}
    E = m c^2%
    \tagaddtext{[\si{\joule}]}
\end{equation}
\item Afkortingen:
\begin{itemize}
    \item  \gls{iot}      % outputs (if first occurance): Internet-of-Things (IoT)
          % outputs (else): IoT
    \item \glspl{lpwan}  % outputs (if first occurance): Low-Power Wide-Area Networks (LPWANs)
          % outputs (else): LPWANs
    \item \acrlong{iot}  % outputs: Internet-of-Things
    \item \acrshort{iot} % outputs: IoT
    \item \acrfull{iot}  % outputs: Internet-of-Things (IoT)
\end{itemize}
\item \verb!\ce{CO2}! \ce{CO2}
\item Refs:
\begin{itemize}

    \item \cref{fig:example}
    \item \cref{tab:example}
    \item \cref{ch:introduction}
    \item \cref{eq:emc2}


    \item \verb!\cref{fig:example}! becomes \cref{fig:example}
    \item \verb!\cref{tab:example}! becomes \cref{tab:example}
    \item \verb!\cref{ch:introduction}! becomes \cref{ch:introduction}
    \item \cref{eq:emc2}


\end{itemize}
\item numerical citation~\cite{s19030585}
\begin{itemize}
    \item \verb!\cite{CallebautGilles2019ByoS}! becomes \cite{CallebautGilles2019ByoS}

    \item \verb!\citeauthor{CallebautGilles2019ByoS}! becomes \citeauthor{CallebautGilles2019ByoS}
    \item \verb!\citeauthorref{CallebautGilles2019ByoS}! becomes \citeauthorref{CallebautGilles2019ByoS}
    \item \verb!\autocite[chap.~2]{CallebautGilles2019ByoS}! becomes \autocite[chap.~2]{CallebautGilles2019ByoS}
    \item \verb!\footcite{CallebautGilles2019ByoS}! becomes \footcite{CallebautGilles2019ByoS}
    \item \verb!\fullcite{CallebautGilles2019ByoS}! becomes \fullcite{CallebautGilles2019ByoS}
\end{itemize}

    \item A $\SI{45}{\degree}$ angle or a \ang{45}.

    \item It is \SI{17}{\degreeCelsius} outside.

    \item  \num{10000}
    \item \num{3.45d-4}

    \item \si{\kilo\gram\meter\per\square\second} 
    \item \si{kg.m/s^2} %unit only

    \item \SI{10}{\percent}
    \item \SI{68}{kg}
\end{itemize}


\begin{center}
    \begin{tabular}{ l  l  l  p{5cm}}
        \hline
        Day       & Min Temp & Max Temp & Summary                                                      \\ \hline
        Monday    & 11C      & 22C      & A clear day with lots of sunshine.
        However, the strong breeze will bring down the temperatures.                                   \\ \hline
        Tuesday   & 9C       & 19C      & Cloudy with rain, across many northern regions. Clear spells
        across most of Scotland and Northern Ireland,
        but rain reaching the far northwest.                                                           \\ \hline
        Wednesday & 10C      & 21C      & Rain will still linger for the morning.
        Conditions will improve by early afternoon and continue
        throughout the evening.                                                                        \\
        \hline
    \end{tabular}
\end{center}







\newcolumntype{R}{>{\raggedleft\arraybackslash}X}%


\begin{center}
    \begin{tabularx}{\textwidth}{ cccX }
        \hline
        Day       & Min Temp & Max Temp & Summary                                                      \\ \hline
        Monday    & 11C      & 22C      & A clear day with lots of sunshine.
        However, the strong breeze will bring down the temperatures.                                   \\ \hline
        Tuesday   & 9C       & 19C      & Cloudy with rain, across many northern regions. Clear spells
        across most of Scotland and Northern Ireland,
        but rain reaching the far northwest.                                                           \\ \hline
        Wednesday & 10C      & 21C      & Rain will still linger for the morning.
        Conditions will improve by early afternoon and continue
        throughout the evening.                                                                        \\
        \hline
    \end{tabularx}
\end{center}



\begin{figure}
    \centering
    \includegraphics[width=0.95\linewidth]{example.jpg}
    \caption{Example JPG}%
    \label{fig:example}
\end{figure}

\begin{table}[ht]
    \centering
    \caption{Fixed-width columns.}%
    \label{tab:example}
    \begin{tabular}[t]{l>{\raggedright}p{0.3\linewidth}>{\raggedright\arraybackslash}p{0.3\linewidth}}
        \toprule
                     & Treatment A                    & Treatment B                     \\
        \midrule
        John Smith   & Good response, no side-effects & No response                     \\
        Jane Doe     & --                             & Good response, no side-effects  \\
        Mary Johnson & No response                    & Good response with side-effects \\
        \bottomrule
    \end{tabular}
\end{table}%



\begin{listing}[ht]
    \inputminted{python}{code/example.py}
    \caption{Minimal working example}
    \label{listing:1}
\end{listing}



%% !TeX root = thesis.tex

%second chapter of your thesis
\chapter{Bespreking}
In het vorig hoofdstuk hebben we naar deze tekst verwezen\label{verwijzing}.

%% !TeX root = thesis.tex

%%%%%%%%%%%%%%%%%%%%%%%%%%%%%%%%%%%%%%%%%%%%%%%%%%%%%%%%%%%%%%%%%%% 
%                                                                 %
%                            CHAPTER                              %
%                                                                 %
%%%%%%%%%%%%%%%%%%%%%%%%%%%%%%%%%%%%%%%%%%%%%%%%%%%%%%%%%%%%%%%%%%% 
 
\chapter{This is the another Chapter}
 
You can say great work has been done about something \citep{Castleman98,Granlund95} or say that \citet{Holmes95} did something really great.
xxxx xxxxx xxxx xxxxxxxxx 
xxx xxxxx xxxxx xxx xxxx xxxx xxxxx xxxxx xxxx xxxxx xxxx xxxxxxxxx
 
\begin{figure}
\vspace{2.0in}
\caption{This is the Caption for Figure 1}
\end{figure}
 
xxx xxxxx xxxxx xxx xxxx xxxx xxxxx xxxxx xxxx xxxxx xxxx xxxxxxxxx
xxx xxxxx xxxxx xxx xxxx xxxx xxxxx xxxxx xxxx xxxxx xxxx xxxxxxxxx
xxx xxxxx xxxxx xxx xxxx xxxx xxxxx xxxxx xxxx xxxxx xxxx xxxxxxxxx
 
\begin{table}[t]
\begin{center}
\begin{tabular}{lll}
Here's       & an          & example  \\
of           & a           & table    \\
floated      & with        & the      \\
\verb+table+ & environment & command.
\end{tabular}
\end{center}
\caption{This is the Caption for Table 1}
\end{table}
 
xxx xxxxx xxxxx xxx xxxx xxxx xxxxx xxxxx xxxx xxxxx xxxx xxxxxxxxx
xxx xxxxx xxxxx xxx xxxx xxxx xxxxx xxxxx xxxx xxxxx xxxx xxxxxxxxx
 
\section{This is a Section Heading}
 

% Bibliografie: referenties. De items zitten in bib.bib
%%%%%%%%%%%%%%%%%%%%%%%%%%%%%%%%%%%%%%%%%%%%%%%%%%%%%%%%%%%%%%%%%
\printbibliography


% Eventueel enkele appendices
%%%%%%%%%%%%%%%%%%%%%%%%%%%%%%
\appendix
\input{chapters/bijlage1}

% Bijlage met daarin het wetenschappelijk artikel
%%%%%%%%%%%%%%%%%%%%%%%%%%%%%%%%%%%%%%%%%%%%%%%%%%
\chapter{Beschrijving van deze masterproef in de vorm van een wetenschappelijk artikel}
The thesis should also contain a short scientific article. If you write your thesis in Dutch, you must write the article in English, and vice versa. We advise you to employ the IEEE Manuscript Templates for Conference Proceedings (\url{https://www.overleaf.com/latex/templates/ieee-conference-template/grfzhhncsfqn}).
Compile the article in another project and include the generated pdf file as shown below:

\includepdf{article.pdf}

% Bijlage met daarin de poster
%%%%%%%%%%%%%%%%%%%%%%%%%%%%%%%
\chapter{Poster}
%\includepdf{poster.pdf}

\chapter{GenAI code of conduct for students}

Generative AI (GenAI) assistance tools can be used to generate text, image, code, video, music, or combinations of these. It includes typical tools like (but this list is not limited to): ChatGPT, Google Gemini, MS Copilot, Midjourney, Claude.ai, Perplexity.ai, Dall-E, etc.

\section*{Student Information}
\begin{tabbing}
\makeatletter% required keep
Student name: \@forenameA \ \@surnameA
\makeatother
\end{tabbing}

This form is related to my Master thesis. \\

Title Master thesis: \underline{\hspace{5cm}} \hspace{1cm} Promoter: \underline{\hspace{5cm}} \\
Daily supervisor: \underline{\hspace{5cm}}

\section*{Use of GenAI Assistance}
Please indicate with "X" (possibly multiple times) in which way you were using GenAI:

\begin{itemize}
    \item[$\Box$] I did not use any GenAI assistance tool.
    \item[$\Box$] I did use GenAI assistance. In this case, specify which ones (e.g., ChatGPT, \ldots): \underline{\hspace{5cm}}
\end{itemize}

\section*{GenAI Assistance Used As/For}
\begin{itemize}
    \item[$\Box$] MS Copilot \\
    \item[$\Box$] Language assistance \\
    \item[$\Box$] Search engine \\
    \item[$\Box$] Literature search \\
    \item[$\Box$] Short-form input assistance \\
    \item[$\Box$] Generating programming code \\
    \item[$\Box$] Generating new research ideas \\
    \item[$\Box$] Generating blocks of text \\
    \item[$\Box$] Other (specify): \underline{\hspace{5cm}}
\end{itemize}

\section*{Code of Conduct for Different Uses}
\subsection*{Language Assistant for Reviewing or Improving Texts}
This use is similar to using spelling and grammar check tools, and you do not have to refer to the use of GenAI in the text. Be careful:
\begin{itemize}
    \item Using GenAI tools on texts you did not write yourself to cover up plagiarism (by paraphrasing original texts) is not allowed.
\end{itemize}

\subsection*{Search Engine for Information or Existing Research}
This use is similar to a Google search or checking Wikipedia. If you then write your own text based on this information, you do not have to refer to the use of GenAI in the text. Be careful:
\begin{itemize}
    \item Be aware that the output of the GenAI tool cannot be guaranteed as a 100\% reliable source of information.
    \item The output may not be entirely correct and is limited due to the databases it uses. Knowledge evolves and may change over time, and it may be that the database of the GenAI tool is not up to date.
\end{itemize}

\subsection*{Literature Search}
This use is comparable to a Google Scholar search. Be careful:
\begin{itemize}
    \item Be aware that the output is restricted to the database it is built on. After this initial search, look for scientific sources and conduct your own analysis.
    \item GenAI tools (like ChatGPT) may output no or wrong references. As a student, you are responsible for further checking and verifying the accuracy of references.
\end{itemize}

\subsection*{Short-form Input Assistance}
This use is similar to Google Docs powered by generative language models.

\subsection*{Generating Programming Code}
The use of GenAI for coding should be explicitly allowed by the teacher. If used for coding, correctly mention the use of GenAI assistance and cite it.

\subsection*{Generating New Research Ideas}
Further verify whether the idea is novel or not. It is likely that it is related to existing work, which should be referenced.

\subsection*{Generating Blocks of Text}
Inserting blocks of text without quotes and a reference to GenAI assistance in your report or thesis is not allowed. Be careful:
\begin{itemize}
    \item When you literally copy elements from a conversation with a GenAI tool, quote them between quotation marks and refer to them according to the specified reference style.
    \item Describe the use of the GenAI tool (tool name, version, date, etc.) in the method section and optionally add the full conversation as an attachment.
\end{itemize}

\subsection*{Other}
Contact the professor of the course or the promoter of the thesis. Inform also the program director. Motivate how you comply with Article 84 of the exam regulations. Explain the use and added value of ChatGPT or another AI tool.

\section*{Further Important Guidelines and Remarks}
\begin{itemize}
    \item GenAI assistance cannot be used related to data or subjects under Non-Disclosure Agreement.
    \item GenAI assistance cannot be used related to sensitive or personal data due to privacy issues.
    \item Take a scientific and critical attitude when interacting with GenAI assistance and interpreting its output.
    \item As a student, you are responsible for complying with Article 84 of the exam regulations: your report or thesis should reflect your own knowledge, understanding, and skills.
\end{itemize}

\subsection*{Exam Regulations Article 84}
``Every conduct individual students display with which they (partially) inhibit or attempt to inhibit a correct judgement of their own knowledge, understanding and/or skills or those of other students, is considered an irregularity which may result in a suitable penalty. A special type of irregularity is plagiarism, i.e., copying the work (ideas, texts, structures, designs, images, plans, codes, \ldots) of others or prior personal work in an exact or slightly modified way without adequately acknowledging the sources.''

\subsection*{Additional Reading}
More information about being transparent on the use of GenAI assistance and about citing and referencing GenAI can be found on the student website.

\subsection*{A Few Final Words}
If you are uncertain whether or not you should declare your use of AI tools, discuss the matter with your teacher or promoter. It is safer to declare AI use when it is not needed than to withhold that declaration when it is required.

Finally, remember that advanced AI tools are new and that they can do things they could not do until recently. It is important to follow up on the most recent developments in AI technologies and communicate openly with your teachers, assistants, supervisors, and peers.



\end{document}
